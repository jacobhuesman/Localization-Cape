\documentclass[]{book}

\usepackage[utf8]{inputenc}
\usepackage{amsmath}
\usepackage{amsfonts}
\usepackage{amssymb}
\usepackage{graphicx}
\usepackage[left=1.00in, right=1.00in, top=1.00in, bottom=1.00in]{geometry}

\usepackage{parskip}
\usepackage{enumitem}
\usepackage[table]{xcolor}
\usepackage{hyperref}

\hypersetup{
	colorlinks=true,
	hidelinks=true,
	breaklinks=true,
	linkcolor=blue,
	filecolor=magenta,      
	colorlinks=false,
	pdftitle={Localization Cape Manual},
	bookmarks=true,
	pdfpagemode=FullScreen,
}
\def\UrlBreaks{\do\/\do-}

%opening
\title{Localization Cape Manual}
\author{Jacob Huesman}

\begin{document}
{
	\let\cleardoublepage\clearpage
	\maketitle
	\tableofcontents
}

\chapter{Localization Cape Requirements}
\section{Power Requirements}
\subsection{Pocket Beagle}
Should be the same as the BeagleBone Black. Minimum recommended is 5V @ 1.2A, recommended is 5V @ 2A, which includes room for USB peripherals. See: \url{http://beagleboard.org/support/faq}

The actual usage by the BeagleBone Black measured during a test done by Adafruit saw current usage of less than 500mA. Which makes sense since it can operate as a USB device. See: \url{https://learn.adafruit.com/embedded-linux-board-comparison/power-usage}. It might actually be lower since there everything is integrated on the pocket beagle.

\subsection{ELP VGA USB Camera Module}
According to the specs available on Amazon this camera doesn't exceed 160mA of power usage. See: \url{https://www.amazon.com/gp/product/B01DRG250Q/ref=oh_aui_search_detailpage?ie=UTF8&psc=1}. Since they are USB Devices we can expect no more than 5V @ 500mA. 

\subsection{Dynamixel XL-320}
See: \url{http://www.robotis.us/dynamixel-xl-320/} and
6~8.4V


\end{document}
